% ****************************************************************************************************
% UCT Typographic Thesis
% Configuration File
% Version 1.0 (4/6/22)
%
% Original author:
% André Miede and Ivo Pletikosić with slight modifications by:
% Matthew Dunk (dnkmat001@myuct.ac.za)
%
% License:
% This work may be distributed and/or modified under the
% conditions of the LaTeX Project Public License, either version 1.3
% of this license or (at your option) any later version.
% The latest version of this license is in
%   http://www.latex-project.org/lppl.txt
% and version 1.3 or later is part of all distributions of LaTeX
% version 2005/12/01 or later.
%
% This work has the LPPL maintenance status `maintained'.
% 
% The Current Maintainer of this work is Matthew J. Dunk
% ****************************************************************************************************


% ****************************************************************************************************
% 0. Set the encoding of your files.
% ****************************************************************************************************
\usepackage[utf8]{inputenc}
\usepackage[T1]{fontenc}

% ****************************************************************************************************
% 1. Configure classicthesis to suit my needs
% ****************************************************************************************************
\PassOptionsToPackage{
	drafting=true,    % print version information on the bottom of the pages
	tocaligned=false, % the left column of the toc will be aligned (no indentation)
	dottedtoc=false,  % page numbers in ToC flushed right
	eulerchapternumbers=true, % use AMS Euler for chapter font (otherwise Palatino)
	linedheaders=false,       % chaper headers will have line above and beneath
	floatperchapter=true,     % numbering per chapter for all floats (i.e., Figure 1.1)
	eulermath=false,  % use awesome Euler fonts for mathematical formulae (only with pdfLaTeX)
	beramono=true,    % toggle a nice monospaced font (w/ bold)
	palatino=true,    % deactivate standard font for loading another one, see the last section at the end of this file for suggestions
	style=classicthesis % classicthesis, arsclassica
}{classicthesis}


% ****************************************************************************************************
% 2. Personal data and user ad-hoc commands (insert your own data here)
% ****************************************************************************************************
\newcommand{\myTitle}{The Stone-Weierstra{\ss} Theorem\xspace}
\newcommand{\mySubtitle}{}
\newcommand{\myDegree}{Honours in Mathematics\xspace}
\newcommand{\myName}{Matthew J. Dunk\xspace}
\newcommand{\mySupervisor}{DR N.R.C Robertson\xspace}
\newcommand{\myDepartment}{Department of Mathematics and Applied Mathematics\xspace}
\newcommand{\myUni}{University of Cape Town\xspace}
\newcommand{\myLocation}{Cape Town\xspace}
\newcommand{\myTime}{October 2022\xspace}
\newcommand{\myVersion}{\myTitle v0.1}

% ********************************************************************
% Setup, finetuning, and useful commands
% ********************************************************************
\providecommand{\mLyX}{L\kern-.1667em\lower.25em\hbox{Y}\kern-.125emX\@}
\newcommand{\ie}{i.\,e.}
\newcommand{\Ie}{I.\,e.}
\newcommand{\eg}{e.\,g.}
\newcommand{\Eg}{E.\,g.}
\newcommand{\vocab}[1]{\textbf{\emph{#1}}}
% ****************************************************************************************************


% ****************************************************************************************************
% 3. Loading some handy packages
% ****************************************************************************************************
% ********************************************************************
% Packages with options that might require adjustments
% ********************************************************************
\usepackage[british]{babel}

\usepackage[babel=true]{csquotes}   % context sensitive quotation facilities

\PassOptionsToPackage{%
	backend=bibtex8,bibencoding=ascii,%
	language=auto,%
	style=numeric-comp,%
	sorting=nyt, % name, year, title
	maxbibnames=10, % default: 3, et al.
	natbib=true, % natbib compatibility mode (\citep and \citet still work)
	doi=false,
	isbn=false
}{biblatex}
\usepackage{biblatex}

\usepackage{mathmacros} % collection of useful mathematics macros


% ********************************************************************
% General useful packages
% ********************************************************************
\usepackage{graphicx}                       % enhanced support for graphics
\usepackage{ragged2e}                       % improved ragged text
\usepackage{scrhack}                        % fix warning when using KOMA with <listings> 
\usepackage{xspace}                         % to get spacing after macros right
\usepackage[british]{isodate}               % date formatting
\usepackage[usenames,svgnames,dvipsnames]{xcolor}
\usepackage{siunitx}                        % unit and number typesetting
\usepackage{array}                          % improved array- and tabular-environments
\usepackage{booktabs}                       % professional table typesetting
\usepackage{keystroke}                      % graphical respresentation of keys on keyboard
\usepackage[printonlyused,smaller]{acronym} % nice macros for handling all acronyms in the thesis
\usepackage[inline]{enumitem}               % customized lists (incompatible(?) with glossaries)

\setlist[enumerate, 1]{label=(\roman*)}

\def\bflabel#1{{\acsfont{#1}\hfill}}
\def\aclabelfont#1{\acsfont{#1}}
% ****************************************************************************************************


% ****************************************************************************************************
% 4. Setup floats: tables, (sub)figures, and captions
% ****************************************************************************************************
\usepackage{tabularx}                   % tabulars with adjustable-width columns
\setlength{\extrarowheight}{3pt}        % increase table row height
\newcommand{\tableheadline}[1]{\multicolumn{1}{l}{\spacedlowsmallcaps{#1}}}
\newcommand{\myfloatalign}{\centering}  % to be used with each float for alignment
\usepackage{subfig}                     % figures broken into subfigures
% ****************************************************************************************************





% ****************************************************************************************************
% 6. Setup code listings
% ****************************************************************************************************
\usepackage{listings}   % Typesetting code
\lstset{language=[LaTeX]Tex,,
	morekeywords={PassOptionsToPackage,selectlanguage},
	keywordstyle=\color{RoyalBlue},
	basicstyle=\small\ttfamily,
	commentstyle=\color{Green}\ttfamily,
	stringstyle=\rmfamily,
	numbers=none,
	numberstyle=\scriptsize,
	stepnumber=5,
	numbersep=8pt,
	showstringspaces=false,
	breaklines=true,
	belowcaptionskip=.75\baselineskip
}   
% ****************************************************************************************************




% ****************************************************************************************************
% 7. Last calls before the bar closes
% ****************************************************************************************************
% ********************************************************************
% Fancy colours from UCT colour palettes webpage
% ********************************************************************
\definecolor{UCTblue}{RGB}{0,144,208}
\definecolor{UCTindigo}{RGB}{89,0,117}


% ********************************************************************
% Load classicthesis itself
% ********************************************************************
\usepackage{classicthesis}

% ********************************************************************
% Setup TikZ (TikZ should be called after classicthesis)
% ********************************************************************
\usepackage{pgfplots}   % external TikZ/PGF Support 
\usepackage{tikz-cd}    % create commutative diagrams with TikZ
\usetikzlibrary{calc,intersections,decorations.markings,matrix,positioning,arrows}
% ********************************************************************
% Fine-tune hyperreferences (hyperref should be called last)
% ********************************************************************
\hypersetup{
	colorlinks=true, linktocpage=true, pdfstartpage=1, pdfstartview=FitV,%
	breaklinks=true, pageanchor=true,%
	pdfpagemode=UseNone,%
	plainpages=false, bookmarksnumbered, bookmarksopen=true, bookmarksopenlevel=1,%
	hypertexnames=true, pdfhighlight=/O,%
	urlcolor=CTurl, linkcolor=CTlink, citecolor=CTcitation,%
	pdftitle={\myTitle},%
	pdfauthor={\textcopyright\ \myName, \myUni, \myDepartment},%
	pdfsubject={},%
	pdfkeywords={},%
	pdfcreator={LuaLaTeX},% 
}  


% ********************************************************************
% Setup autoreferences (hyperref and babel)
% ********************************************************************
% There are some issues regarding autorefnames
% http://www.tex.ac.uk/cgi-bin/texfaq2html?label=latexwords
% you have to redefine the macros for the
% language you use, e.g., american, ngerman
% (as chosen when loading babel/AtBeginDocument)
% ********************************************************************
\makeatletter
\@ifpackageloaded{babel}%
{%
	\addto\extrasbritish{%
		\renewcommand*{\figureautorefname}{Figure}%
		\renewcommand*{\tableautorefname}{Table}%
		\renewcommand*{\partautorefname}{Part}%
		\renewcommand*{\chapterautorefname}{Chapter}%
		\renewcommand*{\sectionautorefname}{Section}%
		\renewcommand*{\subsectionautorefname}{Section}%
		\renewcommand*{\subsubsectionautorefname}{Section}%
	}%
	% Fix to getting autorefs for subfigures right (thanks to Belinda Vogt for changing the definition)
	\providecommand{\subfigureautorefname}{\figureautorefname}%
}{\relax}
\makeatother


% ********************************************************************
% Setup axiom, theorem and theorem-like environments.
% ******************************************************************** 

\usepackage[framemethod=TikZ]{mdframed} % framed environments that can split at page 
% boundaries


\theoremstyle{definition}                                       
\mdfdefinestyle{mdbluebox}{%
	roundcorner = 10pt,
	linewidth=1pt,
	skipabove=12pt,
	innerbottommargin=9pt,
	skipbelow=2pt,
	nobreak=true,
	linecolor=blue,
	backgroundcolor=TealBlue!5,
}
\declaretheoremstyle[
headfont=\sffamily\bfseries\color{MidnightBlue},
mdframed={style=mdbluebox}
]{thmbluebox}

\mdfdefinestyle{mdredbox}{%
	linewidth=0.5pt,
	skipabove=12pt,
	frametitleaboveskip=5pt,
	frametitlebelowskip=0pt,
	skipbelow=2pt,
	frametitlefont=\bfseries,
	innertopmargin=4pt,
	innerbottommargin=8pt,
	nobreak=true,
	linecolor=RawSienna,
	backgroundcolor=Salmon!5,
}
\declaretheoremstyle[
headfont=\bfseries\sffamily\color{RawSienna},
mdframed={style=mdredbox}
]{thmredbox}

\mdfdefinestyle{mdgreenbox}{%
	linewidth=0.5pt,
	skipabove=12pt,
	frametitleaboveskip=5pt,
	frametitlebelowskip=0pt,
	skipbelow=2pt,
	frametitlefont=\bfseries,
	innertopmargin=4pt,
	innerbottommargin=8pt,
	nobreak=true,
	linecolor=green!70!black,
	backgroundcolor=ForestGreen!5
}

\declaretheoremstyle[
headfont=\bfseries\sffamily\color{green!50!black},
mdframed={style=mdgreenbox}
]{thmgreenbox}

\declaretheorem[%
style=thmbluebox,name=Theorem, numberwithin=section]{theorem}
\declaretheorem[style=thmbluebox,name=Lemma,sibling=theorem]{lemma}
\declaretheorem[style=thmbluebox,name=Proposition,sibling=theorem]{proposition}
\declaretheorem[style=thmbluebox,name=Corollary,sibling=theorem]{corollary}
\declaretheorem[style=thmredbox,name=Definition,sibling=theorem]{definition}
\declaretheorem[style=thmgreenbox,name=Example,sibling=theorem]{example}

\mdfdefinestyle{mdblackbox}{%
	skipabove=8pt,
	linewidth=3pt,
	rightline=false,
	leftline=true,
	topline=false,
	bottomline=false,
	linecolor=black,
	backgroundcolor=RedViolet!5!gray!5,
}

\declaretheoremstyle[
headfont=\bfseries\sffamily,
bodyfont=\normalfont,
spaceabove=0pt,
spacebelow=0pt,
mdframed={style=mdblackbox}
]{thmblackbox}

\theoremstyle{theorem}
\declaretheorem[name=Remark,sibling=theorem,style=thmblackbox]{remark}

\newcounter{axiomNum}
\NewDocumentCommand{\axiomMark}{}{}

\NewDocumentEnvironment{axiomata}{ o }
{%
	\par\vspace{1ex}%
	\IfValueTF{#1} {\RenewDocumentCommand{\axiomMark}{}{{\itshape #1}}} {}%
	\setcounter{axiomNum}{0}%
	\noindent\tabularx{\textwidth}{l X r}
}
{%
	\endtabularx
}

\NewDocumentCommand{\axiom}{ m m }{%
	\stepcounter{axiomNum}
	\axiomMark\theaxiomNum. & #1 & #2\tabularnewline
}

% ********************************************************************
% Development Stuff
% ********************************************************************
\listfiles
%\PassOptionsToPackage{l2tabu,orthodox,abort}{nag}
%  \usepackage{nag}
%\PassOptionsToPackage{warning, all}{onlyamsmath}
%  \usepackage{onlyamsmath}


% ****************************************************************************************************
% 7. Further adjustments (experimental)
% ****************************************************************************************************
% ********************************************************************
% Changing the text area
% ********************************************************************
%\areaset[current]{312pt}{761pt} % 686 (factor 2.2) + 33 head + 42 head \the\footskip
%\setlength{\marginparwidth}{7em}%
%\setlength{\marginparsep}{2em}%

% ********************************************************************
% Using different fonts
% ********************************************************************
%\usepackage[oldstylenums]{kpfonts} % oldstyle notextcomp
% \usepackage[osf]{libertine}
%\usepackage[light,condensed,math]{iwona}
%\renewcommand{\sfdefault}{iwona}
%\usepackage{lmodern} % <-- no osf support :-(
%\usepackage{cfr-lm} %
%\usepackage[urw-garamond]{mathdesign} <-- no osf support :-(
%\usepackage[default,osfigures]{opensans} % scale=0.95
%\usepackage[sfdefault]{FiraSans}
% \usepackage[opticals,mathlf]{MinionPro} % onlytext
% ********************************************************************
%\usepackage[largesc,osf]{newpxtext}
%\linespread{1.05} % a bit more for Palatino
% Used to fix these:
% https://bitbucket.org/amiede/classicthesis/issues/139/italics-in-pallatino-capitals-chapter
% https://bitbucket.org/amiede/classicthesis/issues/45/problema-testatine-su-classicthesis-style
% ********************************************************************
% ****************************************************************************************************